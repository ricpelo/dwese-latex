\documentclass[a4paper,11pt]{memoir}
\usepackage[T1]{fontenc}
\usepackage[utf8]{inputenc}
\usepackage{bera}
\usepackage{mathpazo}
\usepackage{eulervm}
%\usepackage{lmodern}
%\usepackage{textcomp}
\usepackage[spanish,es-lcroman,es-noindentfirst]{babel}
\usepackage[pdftex]{color,graphicx}
\usepackage[xparse,minted,skins,breakable]{tcolorbox}
\usepackage[pdftex,unicode,colorlinks]{hyperref}
\usepackage{lipsum}

\chapterstyle{veelo}

\newcommand*{\plogo}{\fbox{$\mathcal{PL}$}}

\newcommand*{\titleGM}{\begingroup % Create the command for including the title page in the document
    \hbox{ % Horizontal box
        \hspace*{0.1\textwidth} % Whitespace to the left of the title page
        \rule{1pt}{\textheight} % Vertical line
        \hspace*{0.05\textwidth} % Whitespace between the vertical line and title page text
        \parbox[b]{0.75\textwidth}{ % Paragraph box which restricts text to less than the width of the page
            {\noindent\scalebox{1.5}{\HUGE\bfseries Desarrollo web}\\[1.5\baselineskip]%
                \scalebox{1.5}{\HUGE\bfseries en entorno servidor}\\[2\baselineskip]}
            {\LARGE \textsc{Ricardo Pérez López}}

            \vspace{0.5\textheight} % Whitespace between the title block and the publisher
            {\noindent\large I.E.S. Doñana \plogo}\\[\baselineskip] % Publisher and logo
        }}
        \endgroup}

\title{Desarrollo web en entorno servidor}
\author{Ricardo Pérez López}

% Un tcolorbox vacío, sólo para usarlo como referencia para que las demás cajas
% usen el mismo contador:
\newtcolorbox[auto counter,number within=section]{contador}{}

% Macro que define una plantilla para crear cajas de listados de código:
\newcommand{\mitcb}[5]{%
\DeclareTCBListing[use counter from=contador]{#1}{ O{\url{code/\thetcbcounter.#2}} O{} }{
    listing engine=minted,
    minted language=#3,
    minted options={fontsize=\small,linenos,numbersep=3mm,#5},
    listing file=code/\thetcbcounter.#2,
    run system command={./linter.sh #3 \thetcbcounter},
    colback=blue!5!white,
    colframe=red!75!black,
    fonttitle=\large\bfseries,
    listing only,
    #4,
    left=8mm,
    enhanced,
    ##2,
    title=Listado~\thetcbcounter: \href{code/\thetcbcounter.#2}{##1},
    before title={\hypersetup{urlcolor=yellow}},
    overlay={\begin{tcbclipinterior}\fill[red!20!blue!20!white] (frame.south west)
            rectangle ([xshift=8mm]frame.north west);\end{tcbclipinterior}}}
}

% Macro que define una plantilla para crear listados de código que se cargan desde
% un archivo:
\newcommand{\miitcb}[5]{%
\DeclareTCBInputListing[use counter from=contador]{#1}{ O{} O{} m }{%
    listing engine=minted,
    minted language=#3,
    minted options={fontsize=\small,linenos,numbersep=3mm,#5},
    listing file=##3,
    run system command={./linter.sh #2 ##3},
    colback=blue!5!white,
    colframe=red!75!black,
    fonttitle=\large\bfseries,
    listing only,
    #4,
    left=8mm,
    enhanced,
    ##2,
    title=Listado~\thetcbcounter: \href{##3}{\ifthenelse{\equal{##1}{}}{##3}{##1}},
    before title={\hypersetup{urlcolor=yellow}},
    overlay={\begin{tcbclipinterior}\fill[red!20!blue!20!white] (frame.south west)
            rectangle ([xshift=8mm]frame.north west);\end{tcbclipinterior}}}
}

% Parámetros:
% #1 : Nombre del entorno
% #2 : Extensión del archivo resultante
% #3 : Lenguaje para minted (o sea, el lexer para Pygmentize)
% #4 : Opciones adicionales para \newtcblisting
% #5 : Opciones adicionales para minted

\mitcb{php}{php}{php}{}{startinline,firstline=3,firstnumber=1}
\mitcb{htmlphp}{php}{html+php}{}{}

% Parámetros:
% #1 : Nombre del comando
% #2 : Tipo de archivo para el linter
% #3 : Lenguaje para minted (o sea, el lexer para Pygmentize)
% #4 : Opciones adicionales para \newtcbinputlisting
% #5 : Opciones adicionales para minted

\miitcb{\iphp}{iphp}{php}{}{startinline,firstline=3,firstnumber=1}
\miitcb{\ihtmlphp}{htmlphp}{html+php}{}{}

\DeclareTotalTColorBox{\cajainfo}{ s O{} O{hint} m }
    {center,width=0.9\linewidth,fonttitle=\large\bfseries,
     IfBooleanTF={#1}
         {colback=green!5}%
         {bicolor,sidebyside,lefthand width=3cm,colback=green!50!black!50,
          colbacklower=green!5},
     #2}
    {\IfBooleanTF {#1}{}{\includegraphics[width=\linewidth]{iconos/#3}\tcblower}%
     \small{#4}}

\begin{document}

% Cubierta

\pagestyle{empty}
\titleGM
\cleardoublepage

% Portada

\thispagestyle{empty}
\large
\hfill
\vfill

\begin{center}
    \color{blue}
    \scalebox{1.5}{\HUGE{Desarrollo web en}}\\[1.3\baselineskip]\scalebox{1.5}{\Huge{entorno servidor}}
\end{center}

\bigskip
\bigskip

\begin{center}
    {\LARGE{}Ricardo Pérez López}
\end{center}

\vfill
\vfill

%\begin{center}
%    D.W.E.S.
%\end{center}
%
%\vfill

\begin{center}
    {\Large I.E.S. Doñana\\[1.1\baselineskip]
    \today}
\end{center}

%%%%%%%%%%%%%%%%%%%%%%%%%%%%%%
% Contenido

\frontmatter
\pagestyle{headings}

\tableofcontents

\chapter{Agradecimientos}

\mainmatter

\part{Conceptos básicos}

\chapter{Introducción}

\section{Prueba}

\cajainfo[title=Prueba]{\lipsum[1]}

\lipsum

\begin{htmlphp}[Prueba][label=etiqueta,breakable]
<html>
    <body><?php
        class Articulos extends CI_Controller
        {
            public function index()
            {
                $this->load->view('articulos/index');
            }
        } ?>
    </body>
</html>
<html>
<body><?php
class Articulos extends CI_Controller
{
public function index()
{
$this->load->view('articulos/index');
}
} ?>
</body>
</html>
<html>
<body><?php
class Articulos extends CI_Controller
{
public function index()
{
$this->load->view('articulos/index');
}
} ?>
</body>
</html>
<html>
<body><?php
class Articulos extends CI_Controller
{
public function index()
{
$this->load->view('articulos/index');
}
} ?>
</body>
</html>
<html>
<body><?php
class Articulos extends CI_Controller
{
public function index()
{
$this->load->view('articulos/index');
}
} ?>
</body>
</html>
<html>
<body><?php
class Articulos extends CI_Controller
{
public function index()
{
$this->load->view('articulos/index');
}
} ?>
</body>
</html>
<html>
<body><?php
class Articulos extends CI_Controller
{
public function index()
{
$this->load->view('articulos/index');
}
} ?>
</body>
</html>
<html>
<body><?php
class Articulos extends CI_Controller
{
public function index()
{
$this->load->view('articulos/index');
}
} ?>
</body>
</html>
\end{htmlphp}

El ref es \ref{etiqueta}, y la página es \pageref{etiqueta}.

\begin{php}[Otra prueba]
class Articulos extends CI_Controller
{
    public function index()
    {
        $this->load->view('articulos/index');
    }
}
\end{php}

\begin{itemize}
  \item pepe
  \item juan
  \item manolo
  \begin{itemize}
    \item pepe
  \end{itemize}
\end{itemize}

\ihtmlphp[][label=otra]{index.php}

Aquí, el ref es \ref{otra} y la página es \pageref{otra}.

\chapter{Otro capítulo}

dfsfsdf

\end{document}
